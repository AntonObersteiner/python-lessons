\documentclass{article}
\usepackage{url}
\usepackage{color}
\definecolor{gray}{rgb}{0.7,0.7,0.7}

\title{PyKurs \\ \normalsize Aufgaben, Quellen, Weiterer Verlauf, etc.}
\author{Anton Obersteiner}

\begin{document}
\maketitle
\paragraph{Quellen}
	Das Material findet sich unter \url{https://github.com/AntonObersteiner/python-lessons/}.
	Die Folien sind dann in \url{/latex/slides/build/}, die Aufgaben in \url{/tasks/}, diese Notiz in \url{/tasks/readme.pdf}

\paragraph{Struktur der Aufgaben}
	Die Aufgaben beginnen mit einer \textbf{Beschreibung}, dann kommt etwas \textbf{unvollständiger Code} und danach meine \textbf{Tests} (Um euch zu sagen, ob die Aufgabe gelöst wurde). Fügt gern eigenen Code ein, mit dem ihr eure Funktionen aufruft um zu sehen, ob sie tun, was sie sollen.
\paragraph{Lösungen}
	Zu einigen Dateien (z.B. \url{/tasks/list/find.py} gibt es Lösungen: \url{/tasks/list/_find.py}). Wenn man gar nicht weiterkommt, kann man da reinschauen, aber eigentlich sind Nachbarn, Kursleitende und das Glossar die besseren Quellen.

\paragraph{Glossar}
	Kurze Zusammenfassung mit Beispielen der bisher besprochenen Themen: \url{/latex/slides/build/glossar.pdf}

\paragraph{Empfohlene Reihenfolge der Übungen} .\\
	\begin{tabular}{l|l}
		/tasks/list/find.py \\
		/tasks/list/sort.py \\
		/tasks/class/Mensch.py \\
		/tasks/dict/calc.py & freie Aufgabenstellung \\
		/tasks/turtle.py & macht damit, was ihr wollt \\
		/tasks/list/primes.py & etwas anspruchsvoller \\
		/tasks/class/Vector.py & Was die meisten schon gemacht haben, aber mit Tests \\
		\textcolor{gray}{/tasks/class/Planet.py} & \textcolor{gray}{Noch nicht fertig, bitte fragen} \\
	\end{tabular}

\paragraph{Themenwünsche}
	Wer ein bestimmtes Thema näher beleuchtet haben möchte, sagt Bescheid. Vorschläge: \\
	\begin{tabular}{ll}
		Datenanalyse/-Visualisierung (matplotlib) &
		Bildverarbeitung (PIL) \\
		Web-Zeug &
		mit Betriebssystem reden? \\
		PyGame (wahrscheinlich sehr advanced) &
		Machine Learning (auch eher advanced) \\
		Nicht Python-spezifisch: & Git, Regex, \LaTeX
	\end{tabular}

\end{document}
