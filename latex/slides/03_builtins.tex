%% Nothing to modify here.
%% make sure to include this before anything else

\documentclass[german]{tudbeamer}
%\usetheme{Szeged}

% packages
\usepackage{color}
\usepackage{listings}

% color definitions
\definecolor{mygreen}{rgb}{0,0.6,0}
\definecolor{mygray}{rgb}{0.5,0.5,0.5}
\definecolor{mymauve}{rgb}{0.58,0,0.82}

% this is needed since the tudbeamer messes things up
\setbeamercolor{title}{fg=white}
\setbeamercolor{subtitle}{fg=white}
\setbeamercolor{supertitle}{fg=white}
\setbeamerfont{supertitle}{series=\bfseries,family=\sffamily}


% re-format the title frame page
\makeatletter
\def\supertitle#1{\gdef\@supertitle{#1}}%
\setbeamertemplate{title page}
{
  \vbox{}
  \vfill
  \begin{centering}
  \begin{beamercolorbox}[sep=8pt,center]{title}
      \usebeamerfont{supertitle}\@supertitle
   \end{beamercolorbox}
    \begin{beamercolorbox}[sep=8pt,center]{title}
    	\usebeamerfont{title}
    	\inserttitle\par%
      	\ifx\insertsubtitle\@empty%
     	\else%
        \vskip0.25em%
        {\usebeamerfont{subtitle}\usebeamercolor[fg]{subtitle}\insertsubtitle\par}%
      	\fi%     
    \end{beamercolorbox}%
    \vskip1em\par
    \begin{beamercolorbox}[sep=8pt,center]{author}
      \usebeamerfont{author}\insertauthor
    \end{beamercolorbox}
    \begin{beamercolorbox}[sep=8pt,center]{institute}
      \usebeamerfont{institute}\insertinstitute
    \end{beamercolorbox}
   \begin{beamercolorbox}[sep=8pt,center]{date}
      \usebeamerfont{date}\insertdate
    \end{beamercolorbox}\vskip0.5em
    {\usebeamercolor[fg]{titlegraphic}\inserttitlegraphic\par}
  \end{centering}
  \vfill
}
\makeatother


% insert frame number
%\expandafter\def\expandafter\insertshorttitle\expandafter{%
%      \insertshorttitle\hfill%
%\insertframenumber\,/\,\inserttotalframenumber}

% preset-listing options
\lstset{
  backgroundcolor=\color{white},   
  % choose the background color; 
  % you must add \usepackage{color} or \usepackage{xcolor}
  basicstyle=\footnotesize,        
  % the size of the fonts that are used for the code
  breakatwhitespace=false,         
  % sets if automatic breaks should only happen at whitespace
  breaklines=true,                 % sets automatic line breaking
  captionpos=b,                    % sets the caption-position to bottom
  commentstyle=\color{mygreen},    % comment style
  % deletekeywords={...},            
  % if you want to delete keywords from the given language
  extendedchars=true,              
  % lets you use non-ASCII characters; 
  % for 8-bits encodings only, does not work with UTF-8
  frame=single,                    % adds a frame around the code
  keepspaces=true,                 
  % keeps spaces in text, 
  % useful for keeping indentation of code 
  % (possibly needs columns=flexible)
  keywordstyle=\color{blue},       % keyword style
  % morekeywords={*,...},            
  % if you want to add more keywords to the set
  numbers=left,                    
  % where to put the line-numbers; possible values are (none, left, right)
  numbersep=5pt,                   
  % how far the line-numbers are from the code
  numberstyle=\tiny\color{mygray}, 
  % the style that is used for the line-numbers
  rulecolor=\color{black},         
  % if not set, the frame-color may be changed on line-breaks 
  % within not-black text (e.g. comments (green here))
  stepnumber=1,                    
  % the step between two line-numbers. 
  % If it's 1, each line will be numbered
  stringstyle=\color{mymauve},     % string literal style
  tabsize=4,                       % sets default tabsize to 4 spaces
  title=\lstname                   
  % show the filename of files included with \lstinputlisting; 
  % also try caption instead of title
}

% macro for code inclusion
\newcommand{\includecode}[2][c]{
	\lstinputlisting[caption=#2, style=custom#1]{#2}
}	% nothing to do here
% TODO change "course_info" to the name of your actual …_info(.tex)
%% Fill in metadata here that do not change over the course
%% They all are marked with the term "TODO".
%% Search functions usually do the trick

% TODO select the targeted language
% Select neither when using tudbeamer
%\usepackage[english]{babel}
\usepackage[ngerman]{babel}

% TODO select the encoding
\usepackage[utf8]{inputenc}
% usepackage[latin1]{inputenc}

\newcommand{\course}{
	Einführung in Python
}

\author{
	Felix Döring
}

\lstset{
	% TODO adapt these settings to your mainly used language
	% also see http://en.wikibooks.org/wiki/LaTeX/Source_Code_Listings
	% NOTE you can override these settings in individual cases
	language = Python,
	showspaces = false,
	showtabs = false,
	showstringspaces = false,
	escapechar = @
}

%% User defined macros here

% Does not work in tables! You have to use \lstinline$...$ instead!
\newcommand{\codeline}[1]{\colorbox{codegray}{\lstinline$#1$}}

% define my own colors
\definecolor{codegray}{gray}{0.97}
\definecolor{stringgreen}{rgb}{0.0, 0.7, 0.6}
 % TODO modify this if you have not already done so

% meta-information
\newcommand{\topic}{
	% TODO fill in the actual topic
	Builtin Datenstrukturen
}

% nothing to do here
\title{\topic}
\supertitle{\course}
\date{\today}

% the actual document
\begin{document}

\maketitle

\begin{frame}
	\tableofcontents
\end{frame}

\section{exceptions}
\begin{frame}
\begin{itemize}
	\item Alle exceptions erben von \codeline{Exception}
	\item catching mit try/except
	\item finally um code auszuführen der unbedingt laufen muss, egal ob eine excetion vorliegt oder nicht
\end{itemize}
\lstinputlisting{resources/03_builtins/exceptions.py}
\end{frame}

\section{booleans}
\begin{frame}
\begin{itemize}
	\item type Name \codeline{bool}
	\item \codeline{True} oder \codeline{False}
	\item operationen sind und, oder, nicht \codeline{and, or, not}
\end{itemize}
\end{frame}

\section{list}
\begin{frame}{list}

\begin{itemize}
	\item enthält variable Anzahl anderer Objekte in gleichbleibender Reihenfolge.
	\item optimiert für einsietige Benutzung wie eine Queue (append und pop)
\end{frame}
\lstinpulisting{resources/03_builtins/list.py}

\section{tuple}
\begin{frame}{tuple}
\begin{itemize}
	\item Gruppiert Daten
	\item kann nicht mehr verändert werden sobald kreiert
\end{itemize}
\lstinputlisting{resources/03_builtins/tuple.py}
\end{frame}

\section{dict}
\begin{frame}{dict}
\begin{itemize}
	\item einfache HashMap
	\item ungeordnet
	\item jeder hashbare Typ kann ein Key sein
	\item dazu zählen Typen selbst, da diese per Default hashbar sind
\end{itemize}
\lstinputlisting{resources/03_builtins/dict.py}
\end{frame}

\section{set/frozenset}
\begin{frame}{set/frozenset}
\begin{itemize}
	\item nur hashbare Einträge
	\item enhält jedes Element nur einmal
	\item schnelle \codeline{in} überprüfung (enthält Element)
	\item Kann operationen wie superset, subset, isdisjoint und difference \codeline{<, >, disdisjoint(), -}
	\item ungeordnet
	\item sets sind nicht hashbar
	\item frozensets sind nicht veränderbar, aber hashbar
	\item -> sets (und frozensets) können frozensets enthalten (da hashbar)
\end{itemize}
\lstinputlisting{resources/03_builtins/set.py}
\end{frame}

\section{iteraton}
\begin{frame}
\begin{itemize}
	\item nur foreach
	\item für Interger iteration gibt es \codeline{range([start], stop, step=1)}
	\item zum iterator kombinieren kann man \codeline{zip(iterator_1, iterator_2, ..., iterator_n)} verwenden
	\item alles mit einer \codeline{__iter__} methode ist iterierbar
	\item \codeline{iter(iterable)} construiert dir einen stateful iterator
\end{itemize}
\lstinputlisting{resources/03_builtins/iterate.py}
\end{frame}

\section{unpacking}
\begin{frame}
\begin{itemize}
	\item kann auf Tupel und Listen angewendet werden
	\item nützlich in for Schleifen
\end{itemize}
\lstinputlisting{resources/03_builtins/unpacking.py}
\end{frame}

\section{contenxt manager}
\begin{frame}
\begin{itemize}
	\item keyword \codeline{with}
	\item kann jedes objekt sein, was eine \codeline{__enter__} und \codeline{__exit__} methode hat
\end{itemize}
\lstinputlisting{resources/03_builtins/cm.py}
\end{frame}

\section{filehandling}
\begin{frame}
\begin{itemize}
	\item open files with \codeline{open(filename, mode="r")}
	\item filehandlers are iterators over lines
\end{itemize}
\lstlistings{resources/03_builtins/file.py}
\end{frame}

% nothing to do from here on
\end{document}
