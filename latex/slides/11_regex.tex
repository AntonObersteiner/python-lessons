\documentclass[ignorenonframetext,]{beamer}
\setbeamertemplate{caption}[numbered]
\setbeamertemplate{caption label separator}{:}
\setbeamercolor{caption name}{fg=normal text.fg}
\usepackage{amssymb,amsmath}
\usepackage{ifxetex,ifluatex}
\usepackage{fixltx2e} % provides \textsubscript
\usepackage{lmodern}
\ifxetex
  \usepackage{fontspec,xltxtra,xunicode}
  \defaultfontfeatures{Mapping=tex-text,Scale=MatchLowercase}
  \newcommand{\euro}{€}
\else
  \ifluatex
    \usepackage{fontspec}
    \defaultfontfeatures{Mapping=tex-text,Scale=MatchLowercase}
    \newcommand{\euro}{€}
  \else
    \usepackage[T1]{fontenc}
    \usepackage[utf8]{inputenc}
      \fi
\fi
% use upquote if available, for straight quotes in verbatim environments
\IfFileExists{upquote.sty}{\usepackage{upquote}}{}
% use microtype if available
\IfFileExists{microtype.sty}{\usepackage{microtype}}{}

% Comment these out if you don't want a slide with just the
% part/section/subsection/subsubsection title:
\AtBeginPart{
  \let\insertpartnumber\relax
  \let\partname\relax
  \frame{\partpage}
}
\AtBeginSection{
  \let\insertsectionnumber\relax
  \let\sectionname\relax
  \frame{\sectionpage}
}
\AtBeginSubsection{
  \let\insertsubsectionnumber\relax
  \let\subsectionname\relax
  \frame{\subsectionpage}
}

\setlength{\parindent}{0pt}
\setlength{\parskip}{6pt plus 2pt minus 1pt}
\setlength{\emergencystretch}{3em}  % prevent overfull lines
\providecommand{\tightlist}{%
  \setlength{\itemsep}{0pt}\setlength{\parskip}{0pt}}
\setcounter{secnumdepth}{0}

\title{re - Regular Expressions}
\date{}

\begin{document}
\frame{\titlepage}

\section{Syntax}\label{syntax}

\begin{frame}

Wie auch bei anderen Sprachen, gibt es in Python regular expressions.
Dafür gibt es einige Sonderzeichen:\\
\textbf{.} (Punkt) Matcht alle Zeichen außer
\texttt{\textbackslash{}n}\\
\textbf{\^{}} Markiert den Anfang des Stings\\
\textbf{\$} Markiert das Ende des Strings (oder das Zeilenende)\\
\textbf{*} Der vorangestellte Charakter muss 0 - n Mal vorkommen\\
\textbf{+} Der vorangestellte Charakter muss 0 - n Mal vorkommen

\end{frame}

\begin{frame}

\textbf{?} Der vorangestellte Charakter muss 0 - 1 Mal vorkommen\\
\textbf{\{m\}} Der vorangestellte Charakter muss genau m Mal vorkommen\\
\textbf{\{m,n\}} Der vorangestellte Charakter muss m - n Mal vorkommen\\
\textbf{\textbackslash{}} zum escapen von Sonderzeichen\\
\textbf{{[}{]}} matcht eines der in den Klammern stehenden Zeichen\\
\textbf{\textbar{}} ist ein ODER entweder der Character davor oder der
danach.

\end{frame}

\begin{frame}

Des weiteren gibt es noch spezielle Sequenzen, wie z. B.\\
\textbf{\textbackslash{}d} für Unicode Ziffern, äquivalent für
\texttt{{[}0-9{]}}\\
\textbf{\textbackslash{}D} ist das Gegenteil, alles was keine Unicode
Ziffern sind\\
\textbf{\textbackslash{}s} für alle Whitespace Charaktere, das
entspricht {[}
\textbackslash{}t\textbackslash{}n\textbackslash{}r\textbackslash{}f\textbackslash{}v{]}\\
\textbf{\textbackslash{}S} entspricht wieder dem Gegenteil\\
\textbf{\textbackslash{}w} für alle Unicode Zeichen {[}a-zA-Z0-9\_{]}\\
\textbf{\textbackslash{}W} für alle Nicht-Unicode Zeichen

\end{frame}

\section{Methoden}\label{methoden}

\begin{frame}[fragile]

\begin{verbatim}
compile(pattern, flags=0)
\end{verbatim}

Wandelt einen String in ein regular expression Objekt um

\begin{verbatim}
search(pattern, string, flags=0)
\end{verbatim}

Sucht in \texttt{string} nach dem Pattern \texttt{pattern}

\begin{verbatim}
match(pattern, string, flags=0)
\end{verbatim}

Sucht am Begin des Strings nach dem Pattern

\begin{verbatim}
fullmatch(pattern, string, flags=0)
\end{verbatim}

Der komplette String und das Pattern müssen übereinstimmen.

\end{frame}

\begin{frame}[fragile]

\begin{verbatim}
findall(pattern, string, flags=0)
\end{verbatim}

Gibt eine Liste von Strings mit allen passenden Übereinstimmungen zurück

\begin{verbatim}
finditer(pattern, string, flags=0)
\end{verbatim}

Gibt einen Iterator, welcher \texttt{match} Objekte beinhaltet zurück

Die restlichen Funktionen können in den
\href{https://docs.python.org/3/library/re.html}{Docs} gefunden werden.

\end{frame}

\section{regular expression Objekt}\label{regular-expression-objekt}

\begin{frame}

Ein solches Objekt hat im Großen und Ganzen die selben Methoden, jedoch
ohne zusaätzliches Pattern, da das Objekt an sich bereits ein Pattern
enthält.

\end{frame}

\section{Match Objekt}\label{match-objekt}

\begin{frame}[fragile]

\begin{verbatim}
start([group])
\end{verbatim}

Gibt die Startposition des Patterns im String zurück

\begin{verbatim}
start([group])
\end{verbatim}

Gibt die Endposition des Patterns im String zurück

\begin{verbatim}
span([group])
\end{verbatim}

Gibt ein Tuple zurück \texttt{(m.start({[}group{]}),m.end({[}group{]}))}

\end{frame}

\end{document}
