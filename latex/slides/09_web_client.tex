
%%	Getting started:
%%	1) Copy this file and name the copy it like the topic it covers
%%	2) In the following make sure to modify the copy, NOT THE ORIGINAL
%%	3) Look for the points marked with "TODO" and complete them
%%	4) compile this file

%% Nothing to modify here.
%% make sure to include this before anything else

\documentclass[german]{tudbeamer}
%\usetheme{Szeged}

% packages
\usepackage{color}
\usepackage{listings}

% color definitions
\definecolor{mygreen}{rgb}{0,0.6,0}
\definecolor{mygray}{rgb}{0.5,0.5,0.5}
\definecolor{mymauve}{rgb}{0.58,0,0.82}

% this is needed since the tudbeamer messes things up
\setbeamercolor{title}{fg=white}
\setbeamercolor{subtitle}{fg=white}
\setbeamercolor{supertitle}{fg=white}
\setbeamerfont{supertitle}{series=\bfseries,family=\sffamily}


% re-format the title frame page
\makeatletter
\def\supertitle#1{\gdef\@supertitle{#1}}%
\setbeamertemplate{title page}
{
  \vbox{}
  \vfill
  \begin{centering}
  \begin{beamercolorbox}[sep=8pt,center]{title}
      \usebeamerfont{supertitle}\@supertitle
   \end{beamercolorbox}
    \begin{beamercolorbox}[sep=8pt,center]{title}
    	\usebeamerfont{title}
    	\inserttitle\par%
      	\ifx\insertsubtitle\@empty%
     	\else%
        \vskip0.25em%
        {\usebeamerfont{subtitle}\usebeamercolor[fg]{subtitle}\insertsubtitle\par}%
      	\fi%     
    \end{beamercolorbox}%
    \vskip1em\par
    \begin{beamercolorbox}[sep=8pt,center]{author}
      \usebeamerfont{author}\insertauthor
    \end{beamercolorbox}
    \begin{beamercolorbox}[sep=8pt,center]{institute}
      \usebeamerfont{institute}\insertinstitute
    \end{beamercolorbox}
   \begin{beamercolorbox}[sep=8pt,center]{date}
      \usebeamerfont{date}\insertdate
    \end{beamercolorbox}\vskip0.5em
    {\usebeamercolor[fg]{titlegraphic}\inserttitlegraphic\par}
  \end{centering}
  \vfill
}
\makeatother


% insert frame number
%\expandafter\def\expandafter\insertshorttitle\expandafter{%
%      \insertshorttitle\hfill%
%\insertframenumber\,/\,\inserttotalframenumber}

% preset-listing options
\lstset{
  backgroundcolor=\color{white},   
  % choose the background color; 
  % you must add \usepackage{color} or \usepackage{xcolor}
  basicstyle=\footnotesize,        
  % the size of the fonts that are used for the code
  breakatwhitespace=false,         
  % sets if automatic breaks should only happen at whitespace
  breaklines=true,                 % sets automatic line breaking
  captionpos=b,                    % sets the caption-position to bottom
  commentstyle=\color{mygreen},    % comment style
  % deletekeywords={...},            
  % if you want to delete keywords from the given language
  extendedchars=true,              
  % lets you use non-ASCII characters; 
  % for 8-bits encodings only, does not work with UTF-8
  frame=single,                    % adds a frame around the code
  keepspaces=true,                 
  % keeps spaces in text, 
  % useful for keeping indentation of code 
  % (possibly needs columns=flexible)
  keywordstyle=\color{blue},       % keyword style
  % morekeywords={*,...},            
  % if you want to add more keywords to the set
  numbers=left,                    
  % where to put the line-numbers; possible values are (none, left, right)
  numbersep=5pt,                   
  % how far the line-numbers are from the code
  numberstyle=\tiny\color{mygray}, 
  % the style that is used for the line-numbers
  rulecolor=\color{black},         
  % if not set, the frame-color may be changed on line-breaks 
  % within not-black text (e.g. comments (green here))
  stepnumber=1,                    
  % the step between two line-numbers. 
  % If it's 1, each line will be numbered
  stringstyle=\color{mymauve},     % string literal style
  tabsize=4,                       % sets default tabsize to 4 spaces
  title=\lstname                   
  % show the filename of files included with \lstinputlisting; 
  % also try caption instead of title
}

% macro for code inclusion
\newcommand{\includecode}[2][c]{
	\lstinputlisting[caption=#2, style=custom#1]{#2}
}	% nothing to do here
% TODO change "course_info" to the name of your actual …_info(.tex)
%% Fill in metadata here that do not change over the course
%% They all are marked with the term "TODO".
%% Search functions usually do the trick

% TODO select the targeted language
% Select neither when using tudbeamer
%\usepackage[english]{babel}
\usepackage[ngerman]{babel}

% TODO select the encoding
\usepackage[utf8]{inputenc}
% usepackage[latin1]{inputenc}

\newcommand{\course}{
	Einführung in Python
}

\author{
	Felix Döring
}

\lstset{
	% TODO adapt these settings to your mainly used language
	% also see http://en.wikibooks.org/wiki/LaTeX/Source_Code_Listings
	% NOTE you can override these settings in individual cases
	language = Python,
	showspaces = false,
	showtabs = false,
	showstringspaces = false,
	escapechar = @
}

%% User defined macros here

% Does not work in tables! You have to use \lstinline$...$ instead!
\newcommand{\codeline}[1]{\colorbox{codegray}{\lstinline$#1$}}

% define my own colors
\definecolor{codegray}{gray}{0.97}
\definecolor{stringgreen}{rgb}{0.0, 0.7, 0.6}
 % TODO modify this if you have not already done so

% meta-information
\newcommand{\topic}{
	% TODO fill in the actual topic
	Python als Webclient
}

% nothing to do here
\title{\topic}
\supertitle{\course}
\date{\today}

% the actual document
\begin{document}
\maketitle

\begin{frame}
	\tableofcontents
\end{frame}


\begin{frame}[fragile]{Das Package urllib}
	Das Package \texttt{urllib} ist eine n\"utzliche Sammlung mehrerer Module
	zur Arbeit mit URLs.
\end{frame}


\section{urllib.request}
\begin{frame}[fragile]{urllib.request}
	Das Modul \texttt{urllib.request} enth\"alt Funktionen und Klassen, welche
	beim \"Offnen von URLs (vor allem \"uber HTTP) helfen. \\[.5cm]
	Unterst\"utzt werden:
	\begin{itemize}
		\item verschiedene Authentifizierungsarten
		\item Weiterleitungen
		\item Cookies
		\item und mehr...
	\end{itemize}
\end{frame}

\subsection{\"Offnen einer URL}
\begin{frame}[fragile]{\"Offnen einer URL}
	Das \"offnen einer URL wird \"uber die Funktion \texttt{urlopen()} realisiert:
	\lstinputlisting[lastline=5]{resources/09_web_client/urllib_calls.py}
\end{frame}

\begin{frame}[fragile]{\"Offnen einer URL}
	\begin{description}
		\item[\textbf{url}] ein String f\"ur simple URLs oder ein \texttt{Request} Objekt f\"ur komplexere Anfragen
		\item[\textbf{data}] Daten, die an den Server gesendet werden sollen. \\
		vom Typ \texttt{bytes} oder ein Iterable von \texttt{bytes} Objekten \\[.75cm]
	\end{description}
	\textbf{R\"uckgabewerte}
	\begin{description}
		\item bei URLs mit http-Requests \\
		\hspace*{1cm}\texttt{httplib.client.HTTPResponse} Objekt
		\item bei ftp, file und data \\
		\hspace*{1cm}\texttt{urllib.addinfourl} Objekt
	\end{description}
\end{frame}

\subsection{Request Klasse}
\begin{frame}{Request Klasse}
	Um komplexere Anfragen stellen zu k\"onnen, kann man Request Objekte verwenden:
	\lstinputlisting[firstline=7, lastline=10]{resources/09_web_client/urllib_calls.py}
\end{frame}

\begin{frame}{Request Klasse}
	\begin{description}
		\item[\textbf{url}] muss String mit g\"ultiger URl sein
		\item[\textbf{data}] wie bei \textit{urlopen}
		\item[\textbf{headers}] \texttt{dict} mit \texttt{\{Header-Name : Header-Value, ...\}} oder \\
		\texttt{list} von Tupeln mit \texttt{[(Header-Name, Header-Value), ...]}
		\item[\textbf{method}] String, der Art des HTTP Request angibt (\texttt{HEAD}, \texttt{GET}, \texttt{POST}, ...)
	\end{description}
\end{frame}

\begin{frame}{Beispiel}
	\lstinputlisting{resources/09_web_client/req_example.py} \ \\[.5cm]
	Die Request Klasse kann man Verwenden zum:
	\begin{itemize}
		\item Kontrollieren der gesendeten Header \textit{(z.B. Content-Type oder User-Agent)}
		\item Kontrollieren der Method \texttt{POST}, \texttt{PUT} oder \texttt{HEAD}
		% TODO
	\end{itemize}
\end{frame}

\subsection{HTTPResponse Klasse}
\begin{frame}[fragile]{HTTPResponse Klasse}
	Objekte dieser Klasse werden nicht direkt vom User erstellt.
	\lstinputlisting[firstline=13]{resources/09_web_client/urllib_calls.py} \ \\[.5cm]
	Klasse enth\"alt Funktionen und Variablen wie:
	\begin{itemize}
		\item \texttt{read()} - gibt zur\"uckgelieferten Inhalt zur\"uck
		\item \texttt{getheader()} oder \texttt{getheaders()} liefert einen/alle Header zur\"uck
		\item \texttt{status} gibt den HTML Statuscode zur\"uck
		\item \texttt{version} gibt die HTML Version zur\"uck
	\end{itemize}
\end{frame}


\section{Andere Module}
\begin{frame}[fragile]{Andere Module}
	Das Package \texttt{urllib} enth\"alt au{\ss}erdem folgende Module:
	\begin{itemize}
		\item \texttt{urllib.error} \ \\
			Enth\"alt Exceptions, die von \texttt{urllib.request} geworfen werden.
		\item \texttt{urllib.parse} \ \\
			Zum Parsen von URLs.
		\item \texttt{urllib.robotparse} \ \\
			Zum Parsen der \textit{robots.txt} von Webseiten.
	\end{itemize}
\end{frame}


\section{Das Requests-Modul}
\subsection{Installation}
\begin{frame}{Das Requests Modul}
	Das Requests Modul ist eine gute und f\"ur Menschen verst\"andliche Alternative
	zu \texttt{urllib.requests}, das HTTP Requests vereinfacht. \\[.75cm]

	Das Requests Package l\"asst sich ganz einfach \"uber \textbf{pip} installieren: \\
	\texttt{pip install requests}
\end{frame}

\begin{frame}{Beispiel}
	\lstinputlisting{resources/09_web_client/requests_example.py} \ \\[.5cm]
	Ruft die API-Seite von \textbf{GitHub} auf und authentifiziert sich mit Nutzername und Passwort.
\end{frame}

% nothing to do from here on
\end{document}
