
%%	Getting started:
%%	1) Copy this file and name the copy it like the topic it covers
%%	2) In the following make sure to modify the copy, NOT THE ORIGINAL
%%	3) Look for the points marked with "TODO" and complete them
%%	4) compile this file

%% Nothing to modify here.
%% make sure to include this before anything else

\documentclass[german]{tudbeamer}
%\usetheme{Szeged}

% packages
\usepackage{color}
\usepackage{listings}

% color definitions
\definecolor{mygreen}{rgb}{0,0.6,0}
\definecolor{mygray}{rgb}{0.5,0.5,0.5}
\definecolor{mymauve}{rgb}{0.58,0,0.82}

% this is needed since the tudbeamer messes things up
\setbeamercolor{title}{fg=white}
\setbeamercolor{subtitle}{fg=white}
\setbeamercolor{supertitle}{fg=white}
\setbeamerfont{supertitle}{series=\bfseries,family=\sffamily}


% re-format the title frame page
\makeatletter
\def\supertitle#1{\gdef\@supertitle{#1}}%
\setbeamertemplate{title page}
{
  \vbox{}
  \vfill
  \begin{centering}
  \begin{beamercolorbox}[sep=8pt,center]{title}
      \usebeamerfont{supertitle}\@supertitle
   \end{beamercolorbox}
    \begin{beamercolorbox}[sep=8pt,center]{title}
    	\usebeamerfont{title}
    	\inserttitle\par%
      	\ifx\insertsubtitle\@empty%
     	\else%
        \vskip0.25em%
        {\usebeamerfont{subtitle}\usebeamercolor[fg]{subtitle}\insertsubtitle\par}%
      	\fi%     
    \end{beamercolorbox}%
    \vskip1em\par
    \begin{beamercolorbox}[sep=8pt,center]{author}
      \usebeamerfont{author}\insertauthor
    \end{beamercolorbox}
    \begin{beamercolorbox}[sep=8pt,center]{institute}
      \usebeamerfont{institute}\insertinstitute
    \end{beamercolorbox}
   \begin{beamercolorbox}[sep=8pt,center]{date}
      \usebeamerfont{date}\insertdate
    \end{beamercolorbox}\vskip0.5em
    {\usebeamercolor[fg]{titlegraphic}\inserttitlegraphic\par}
  \end{centering}
  \vfill
}
\makeatother


% insert frame number
%\expandafter\def\expandafter\insertshorttitle\expandafter{%
%      \insertshorttitle\hfill%
%\insertframenumber\,/\,\inserttotalframenumber}

% preset-listing options
\lstset{
  backgroundcolor=\color{white},   
  % choose the background color; 
  % you must add \usepackage{color} or \usepackage{xcolor}
  basicstyle=\footnotesize,        
  % the size of the fonts that are used for the code
  breakatwhitespace=false,         
  % sets if automatic breaks should only happen at whitespace
  breaklines=true,                 % sets automatic line breaking
  captionpos=b,                    % sets the caption-position to bottom
  commentstyle=\color{mygreen},    % comment style
  % deletekeywords={...},            
  % if you want to delete keywords from the given language
  extendedchars=true,              
  % lets you use non-ASCII characters; 
  % for 8-bits encodings only, does not work with UTF-8
  frame=single,                    % adds a frame around the code
  keepspaces=true,                 
  % keeps spaces in text, 
  % useful for keeping indentation of code 
  % (possibly needs columns=flexible)
  keywordstyle=\color{blue},       % keyword style
  % morekeywords={*,...},            
  % if you want to add more keywords to the set
  numbers=left,                    
  % where to put the line-numbers; possible values are (none, left, right)
  numbersep=5pt,                   
  % how far the line-numbers are from the code
  numberstyle=\tiny\color{mygray}, 
  % the style that is used for the line-numbers
  rulecolor=\color{black},         
  % if not set, the frame-color may be changed on line-breaks 
  % within not-black text (e.g. comments (green here))
  stepnumber=1,                    
  % the step between two line-numbers. 
  % If it's 1, each line will be numbered
  stringstyle=\color{mymauve},     % string literal style
  tabsize=4,                       % sets default tabsize to 4 spaces
  title=\lstname                   
  % show the filename of files included with \lstinputlisting; 
  % also try caption instead of title
}

% macro for code inclusion
\newcommand{\includecode}[2][c]{
	\lstinputlisting[caption=#2, style=custom#1]{#2}
}	% nothing to do here
% TODO change "course_info" to the name of your actual …_info(.tex)
%% Fill in metadata here that do not change over the course
%% They all are marked with the term "TODO".
%% Search functions usually do the trick

% TODO select the targeted language
% Select neither when using tudbeamer
%\usepackage[english]{babel}
\usepackage[ngerman]{babel}

% TODO select the encoding
\usepackage[utf8]{inputenc}
% usepackage[latin1]{inputenc}

\newcommand{\course}{
	Einführung in Python
}

\author{
	Felix Döring
}

\lstset{
	% TODO adapt these settings to your mainly used language
	% also see http://en.wikibooks.org/wiki/LaTeX/Source_Code_Listings
	% NOTE you can override these settings in individual cases
	language = Python,
	showspaces = false,
	showtabs = false,
	showstringspaces = false,
	escapechar = @
}

%% User defined macros here

% Does not work in tables! You have to use \lstinline$...$ instead!
\newcommand{\codeline}[1]{\colorbox{codegray}{\lstinline$#1$}}

% define my own colors
\definecolor{codegray}{gray}{0.97}
\definecolor{stringgreen}{rgb}{0.0, 0.7, 0.6}
 % TODO modify this if you have not already done so

% meta-information
\newcommand{\topic}{
	% TODO fill in the actual topic
	Getting started
}

% nothing to do here
\title{\topic}
\supertitle{\course}
\date{\today}

% the actual document
\begin{document}

\maketitle

\begin{frame}
	\tableofcontents
\end{frame}

% own content here

% --------------------------- Über den Kurs -----------------------------------
\section{Über diesen Kurs}
\begin{frame}{Über diesen Kurs}
	\begin{itemize}
    	\item 12 Kurseinheiten
    	\item setzt grundlegende Programmierkentnisse voraus
    	\item Ressourcen
    	\begin{itemize}
    	    \item \href{http://auditorium.inf.tu-dresden.de}{auditorium} %TODO: add link to our group once it exists
	        \item Google (python/python 3 meine frage hier) landet oft in der python 2.7 Doku (Versionsswitcher im Menü)
    	    \item \href{docs.python.org}{offizielle Dokumentation}
    	    \item unsere \href{https://github.com/fsr}{Github-Organisation}
    	\end{itemize}
    	\item Hinweis: SCM's sind hilfreich (\href{https://git-scm.com}{git}, \href{http://mercurial.selenic.com/}{mercurial})
	\end{itemize}
\end{frame}

% ----------------------- Der Python Interpreter ------------------------------
\section{Der Python Interpreter}
\begin{frame}{Der Python Interpreter}
	\begin{itemize}
    	\item Die zwei verbreitet verwendeten Python Versionen sind 2.7 und 3.4, wir werden 3.4 nutzen, weil es cooler ist und bessere Features hat
    	\item Python kann auf \href{http://www.python.org}{hier} heruntergeladen und installiert werden oder mit dem Paketmanager eurer Wahl. (Das Paket sollte \texttt{python3} und \texttt{python3-dev} sein, au\ss{}er unter Arch)
    	\item Python funktioniert besser unter Linux und Mac (ist aber okay unter Windows)
    	\item Den Interpreter startet man mit \texttt{python3} im Terminal oder mit \texttt{Python.exe}
    	\item Der Interpreter stellt die volle Funktionalität von Python bereit, einschließ\ss{}lich dem Erstellen von Klassen und Funktionen
	\end{itemize}
\end{frame}

% --------------------------- Python Scripte ----------------------------------
\section{Python Scripte}
\begin{frame}{Python Scripte}
\begin{description}
   	\item[Editor] empfohlen (das ist was wir im Kurs benutzen)
    \begin{itemize}
        \item \href{https://atom.io}{atom} (weil Github)
        \item \href{http://www.sublimetext.com/3}{Sublime Text 3} (\glqq{}Winrar-free\grqq{})
        \item \href{http://notepad-plus-plus.org}{Notepad++} (free)
        \item \href{https://c9.i}{cloud9} (online, free für open source Projekte)
        \item vim/emacs (free)
    \end{itemize}
    \item[IDEs] Benutzen wir hier nicht, da wir kein Kurs über eine IDE machen, sondern über Python selbst (wir beantworten im Kurs keine Fragen zu IDE Problemen)
   	\begin{itemize}
       	\item \href{https://jetbrains.com/pycharm}{PyCharm} (free + professional für Studenten)
       	\item \href{https://wingware.com/}{Wing} (kostenpflichtig)
   	\end{itemize}
   	\item[Struktur]
   	\begin{itemize}
       	\item Python Scripte sind Textdateien, die auf \texttt{.py} enden
        \item Python Packages sind Ordner mit einer \texttt{\_\_init\_\_.py} Datei (behandeln wir später)
    \end{itemize}
\end{description}
\end{frame}


% ----------------------- Grundlagen der Sprache ------------------------------
\section{Grundlagen der Sprache}
\begin{frame}[fragile]{Grundlagen der Sprache}
    Python ist eine schwach typisierte Scriptsprache (weakly typed scripting language). Es gibt Typen (anders als in JavaSript), aber Variablen haben keine festen Typen.\\

    \textbf{Kommentare:}
    \begin{lstlisting}
		# in python nur einzeilige Kommentare
		
		def my_function(params):
		    """
		    Oder docstrings wie dieser,
		    aber nur zu beginn einer Funktions-
		    oder Klassendefinition
		    """
		    pass
    \end{lstlisting}
\end{frame}

\begin{frame}[fragile]
	\textbf{builtin Datentypen:}\\
	\begin{tabular}{c|l}
		Name & Funktion \\ \hline
		\texttt{object} & Basistyp, alles erbt von \texttt{object} \\
		\texttt{int} & Ganzzahl \glqq{}beliebiger\grqq{} Grö\ss{}e \\
		\texttt{float} & Kommazahl \glqq{}beliebiger\grqq{} Grö\ss{}e \\
		\texttt{bool} & Wahrheitswert (\texttt{True}, \texttt{False})\\
		\texttt{None} & Typ des \texttt{None}-Objektes \\
		\texttt{type} & Grundtyp aller Typen (z.B. \texttt{int} ist eine Instanz von \texttt{int}) \\
		\texttt{list} & standard Liste \\
		\texttt{tuple} & unveränderbares n-Tupel \\
		\texttt{set} & (mathematische) Menge von Objekten \\
		\texttt{frozenset} & unveränderbare (mathematische) Menge von Objekten \\
		\texttt{dict} & Hash-Map \\
	\end{tabular}
\end{frame}


% ------------------------- Das erste Programm --------------------------------
\section{Das erste Programm}
\begin{frame}[fragile]{Das erste Programm}
	Ein simples \glqq{}Hallo Welt\grqq{}-Programm:\\[.5cm]
	\begin{lstlisting}
		def my_function():
		    print('Hallo Welt!')
		
		if __name__ == '__main__':
		    my_function()
	\end{lstlisting}
\end{frame}


% ----------------------- Wichtige Eigenschaften ------------------------------
\begin{frame}[fragile]{Das erste Programm}
	\textbf{Wichtige Eigenschaften:}
	\begin{itemize}
	    \item Keine Semikolons
	    \item Kein geschweiften Klammern für Codeblöcke
	    \item Einrückungen zeigen Codeblöcke an
	    \item Funktionsaufrufe immer mit runden Klammern
	    \item Funktionen definieren mit \codeline{def <funktionsname>([parameter_liste, ...]):}
	    \item Variablen mit der Struktur \codeline{__name__} sind spezielle Werte (gewöhnlich aus \codeline{builtin} oder Methoden von Standardtypen)\\
	      z.B. \codeline{__file__} ist immer der Name des Scriptes (in welchem \codeline{__file__} aufgerufen wird),  
	      \codeline{__builtin__} ist das Modul mit den immer verfügbaren Funktionen und Datentypen wie \codeline{list} oder \codeline{tuple}
	\end{itemize}
\end{frame}


% ------------------------------ Operatoren -----------------------------------
\section{Operatoren}
\begin{frame}[fragile]{Operatoren}
	\begin{description}
	    \item[mathematisch] \codeline{+}, \codeline{-}, \codeline{*}, \codeline{/} 
	    \item[vergleichend] \codeline{<}, \codeline{>}, \codeline{<=}, \codeline{>=}, \codeline{==} (Wert gleich), \codeline{is} (gleiches Objekt/gleiche Referenz)
	    \item[logisch] \codeline{and}, \codeline{or}, \codeline{not}\\ \codeline{(a && b) || (!c)} aus C oder Java entspricht \codeline{(a and b) or not c} in Python
	    \item[bitweise] \codeline{&}, \codeline{\|}, \codeline{\<\<}, \codeline{\>\>}, \codeline{\^} (xor), \codeline{\~} (invertieren)
	    \item[Accessoren] \codeline{.} (für Methoden und Attribute), \codeline{[]} (für Datenstrukturen mit Index)
	\end{description}
\end{frame}


% -------------------------- Namenskonventionen -------------------------------
\section{Namenskonventionen}
\begin{frame}[fragile]{Namenskonventionen}
	\begin{description}
	    \item[\textbf{Klassen}] \textit{PascalCase}, alles direkt zusammen, gro\ss{} beginnend und jedes neue Wort gro\ss{}
	    \item[\textbf{Variablen, Funktionen, Methoden}] \textit{snake\_case}, alles klein und Wörter mit Unterstrich getrennt \\
	    \textbf{Merke:} Da \codeline{-} ein Operator ist, ist es in Namen von Variablen, Funktionen etc. \textbf{nicht} zulässig (damit Python eine Kontextfreie Sprache ist)
	    \item[\textbf{protected Variablen, Funktionen, Methoden}] beginnen mit einem Unterstrich \codeline{\_} oder mit zweien \codeline{\_\_} für private
	    \item[\textbf{Merke}] Python hat kein Zugriffsmanagement. Die Regel mit dem Unterstrich ist nur eine Konvention um zu verhindern, dass andere Teile des Codes nutzen, der eine hohe Wahrscheinlichkeit hat in Zukunft verändert zu werden.
	\end{description}
\end{frame}


% ------------------------- Stings - Grundlagen -------------------------------
\section{Strings}
\subsection{Grundlagen}
\begin{frame}[fragile]{Stings - Grundlagen}
	\begin{itemize}
	    \item Der Typ eines Strings ist \codeline{str}.
	    \item Strings sind in Python immutable (nicht veränderbar). Jede String Operation erzeugt einen neuen String.
	    \item Ein String kann erzeugt werden mit einer Zeichenkette in Anführungszeichen, \codeline{\'\'} oder \codeline{\"\"} (beide sind äquivalent).
	    \item rohe Strings mir dem Präfix \codeline{r}, \codeline{r\"mystring\"} oder \codeline{r\'mystring\'}
	    \item Strings in Python 3 sind UTF-8 encoded.
	\end{itemize}
\end{frame}


% ------------------------- Stings - Verknüpfen -------------------------------
\subsubsection{Verknüpfen}
\begin{frame}[fragile]{Stings - Verknüpfen}
	\begin{itemize}
	    \item Strings können durch Konkatenation verknüpft werden \\
	    \begin{lstlisting}
			'Hallo' + '_' + 'Welt' #  => 'Hallo_Welt'
		\end{lstlisting}  
	    \item Listen, Tupel etc. von Strings können via `str.join` verknüpft werden \\
	    \begin{lstlisting}
			'_'.join(['Hallo', 'Welt']) #  => 'Hallo_Welt'
		\end{lstlisting}
	    Dabei ist der String auf welchem die Methode aufgerufen wird der Separator.
	\end{itemize}
\end{frame}


% ------------------------ Stings - Formatierung ------------------------------
\subsection{Formatierung}
\begin{frame}[fragile]{Strings - Formatierung}
	Wir wollen den String \codeline{'my string 4 vier'} erzeugen.

	\begin{lstlisting}
	# mit `str.format()`  

	'my string {} {}'.format(4, 'vier')
	# in Reihenfolge der argumente

	'my string {number} {name}'.format(name='vier', number=4)`
	# via Name, Reihenfolge egal

	'my string {number} {}'.format('vier', number=4)
	# oder beides kombiniert
	\end{lstlisting}
\end{frame}

\begin{frame}[fragile]{Strings - Formatierung}
	Wir wollen den String \codeline{'my string 4 vier'} erzeugen.
	\begin{lstlisting}
	# und mit dem %-Operator

	'string %d %s' % (4, 'vier')
	# in Reihenfolge

	'string %(number)d %(name)s' % {number:4, name:'vier'}
	# via Name
	\end{lstlisting}
\end{frame}



% nothing to do from here on
\end{document}
