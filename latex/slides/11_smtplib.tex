
%%	Getting started:
%%	1) Copy this file and name the copy it like the topic it covers
%%	2) In the following make sure to modify the copy, NOT THE ORIGINAL
%%	3) Look for the points marked with "TODO" and complete them
%%	4) compile this file

%% Nothing to modify here.
%% make sure to include this before anything else

\documentclass[german]{tudbeamer}
%\usetheme{Szeged}

% packages
\usepackage{color}
\usepackage{listings}

% color definitions
\definecolor{mygreen}{rgb}{0,0.6,0}
\definecolor{mygray}{rgb}{0.5,0.5,0.5}
\definecolor{mymauve}{rgb}{0.58,0,0.82}

% this is needed since the tudbeamer messes things up
\setbeamercolor{title}{fg=white}
\setbeamercolor{subtitle}{fg=white}
\setbeamercolor{supertitle}{fg=white}
\setbeamerfont{supertitle}{series=\bfseries,family=\sffamily}


% re-format the title frame page
\makeatletter
\def\supertitle#1{\gdef\@supertitle{#1}}%
\setbeamertemplate{title page}
{
  \vbox{}
  \vfill
  \begin{centering}
  \begin{beamercolorbox}[sep=8pt,center]{title}
      \usebeamerfont{supertitle}\@supertitle
   \end{beamercolorbox}
    \begin{beamercolorbox}[sep=8pt,center]{title}
    	\usebeamerfont{title}
    	\inserttitle\par%
      	\ifx\insertsubtitle\@empty%
     	\else%
        \vskip0.25em%
        {\usebeamerfont{subtitle}\usebeamercolor[fg]{subtitle}\insertsubtitle\par}%
      	\fi%     
    \end{beamercolorbox}%
    \vskip1em\par
    \begin{beamercolorbox}[sep=8pt,center]{author}
      \usebeamerfont{author}\insertauthor
    \end{beamercolorbox}
    \begin{beamercolorbox}[sep=8pt,center]{institute}
      \usebeamerfont{institute}\insertinstitute
    \end{beamercolorbox}
   \begin{beamercolorbox}[sep=8pt,center]{date}
      \usebeamerfont{date}\insertdate
    \end{beamercolorbox}\vskip0.5em
    {\usebeamercolor[fg]{titlegraphic}\inserttitlegraphic\par}
  \end{centering}
  \vfill
}
\makeatother


% insert frame number
%\expandafter\def\expandafter\insertshorttitle\expandafter{%
%      \insertshorttitle\hfill%
%\insertframenumber\,/\,\inserttotalframenumber}

% preset-listing options
\lstset{
  backgroundcolor=\color{white},   
  % choose the background color; 
  % you must add \usepackage{color} or \usepackage{xcolor}
  basicstyle=\footnotesize,        
  % the size of the fonts that are used for the code
  breakatwhitespace=false,         
  % sets if automatic breaks should only happen at whitespace
  breaklines=true,                 % sets automatic line breaking
  captionpos=b,                    % sets the caption-position to bottom
  commentstyle=\color{mygreen},    % comment style
  % deletekeywords={...},            
  % if you want to delete keywords from the given language
  extendedchars=true,              
  % lets you use non-ASCII characters; 
  % for 8-bits encodings only, does not work with UTF-8
  frame=single,                    % adds a frame around the code
  keepspaces=true,                 
  % keeps spaces in text, 
  % useful for keeping indentation of code 
  % (possibly needs columns=flexible)
  keywordstyle=\color{blue},       % keyword style
  % morekeywords={*,...},            
  % if you want to add more keywords to the set
  numbers=left,                    
  % where to put the line-numbers; possible values are (none, left, right)
  numbersep=5pt,                   
  % how far the line-numbers are from the code
  numberstyle=\tiny\color{mygray}, 
  % the style that is used for the line-numbers
  rulecolor=\color{black},         
  % if not set, the frame-color may be changed on line-breaks 
  % within not-black text (e.g. comments (green here))
  stepnumber=1,                    
  % the step between two line-numbers. 
  % If it's 1, each line will be numbered
  stringstyle=\color{mymauve},     % string literal style
  tabsize=4,                       % sets default tabsize to 4 spaces
  title=\lstname                   
  % show the filename of files included with \lstinputlisting; 
  % also try caption instead of title
}

% macro for code inclusion
\newcommand{\includecode}[2][c]{
	\lstinputlisting[caption=#2, style=custom#1]{#2}
}	% nothing to do here
% TODO change "course_info" to the name of your actual …_info(.tex)
%% Fill in metadata here that do not change over the course
%% They all are marked with the term "TODO".
%% Search functions usually do the trick

% TODO select the targeted language
% Select neither when using tudbeamer
%\usepackage[english]{babel}
\usepackage[ngerman]{babel}

% TODO select the encoding
\usepackage[utf8]{inputenc}
% usepackage[latin1]{inputenc}

\newcommand{\course}{
	Einführung in Python
}

\author{
	Felix Döring
}

\lstset{
	% TODO adapt these settings to your mainly used language
	% also see http://en.wikibooks.org/wiki/LaTeX/Source_Code_Listings
	% NOTE you can override these settings in individual cases
	language = Python,
	showspaces = false,
	showtabs = false,
	showstringspaces = false,
	escapechar = @
}

%% User defined macros here

% Does not work in tables! You have to use \lstinline$...$ instead!
\newcommand{\codeline}[1]{\colorbox{codegray}{\lstinline$#1$}}

% define my own colors
\definecolor{codegray}{gray}{0.97}
\definecolor{stringgreen}{rgb}{0.0, 0.7, 0.6}
 % TODO modify this if you have not already done so

% meta-information
\newcommand{\topic}{
	% TODO fill in the actual topic
	Mails in Python senden
}

% nothing to do here
\title{\topic}
\supertitle{\course}
\date{\today}

% the actual document
\begin{document}

\maketitle

\begin{frame}
	\tableofcontents
\end{frame}


\begin{frame}
	Die folgenden Folien enthalten eine praktische Anleitung zum Senden von Mails in Python.
\end{frame}


\section{Grundlagen: Mails senden}
\subsection{Das Modul \texttt{smtplib}}
\begin{frame}[fragile]{Das Modul \texttt{smtplib}}
	Das Modul \texttt{smtplib} definiert eine SMTP\textbf{*}-Client Session, die genutzt werden kann, um von jedem beliebigen, internetf\"ahigen Ger\"at E-Mails zu verschicken. \\[1cm]
	
	\textbf{*}\textit{SMTP} steht für \textbf{S}imple \textbf{M}ail \textbf{T}ransfer 
	\textbf{P}rotocol und ist das Standard-Protokoll zum E-Mail Versand.
\end{frame}

\begin{frame}[fragile]{Verbundung zum Server}
	Die smtplib kann sich zu einem SMTP-Server verbinden\\[.5cm]
	\lstinputlisting[firstline=3, lastline=3]{resources/11_sendmail/defs.py}
	
	\ \\[.25cm]
	Der Server kann zum einen als einheitlicher \texttt{String} (inkl. Port) oder als \texttt{host} und \texttt{port} angeben werden
\end{frame}

\begin{frame}[fragile]{Login auf dem Server}
	Zum einen muss zuerst eine sichere \texttt{tls}-Verbindung mithilfe von der Methode \texttt{starttls()} hergestellt werden.\\[.25cm]
	Der eigentliche Login erfolgt durch:\\
	\lstinputlisting[firstline=5, lastline=5]{resources/11_sendmail/defs.py}
	\ \\[.25cm]
	Der Parameter \texttt{initial\_response\_ok} kann in unserem Fall vernachlässigt werden.
\end{frame}

\begin{frame}[fragile]{Senden der Mail}
	Das tatsächliche Versenden der Mail funktioniert dann mit folgender Methode:\\[.25cm]
	\lstinputlisting[firstline=7]{resources/11_sendmail/defs.py}
	\ \\[.25cm]
	Jedoch l\"asst sich das ganze auch mit einem \texttt{MIME}-Objekt vereinfachen, auf das später noch eingegangen wird.
\end{frame}

\begin{frame}{Schließen der Verbindung}
	Zum Schluss darf nicht vergessen werden, die Verbindung zum SMTP-Server wieder zu schließen.\\
	Dies geschieht mit der methode \texttt{quit()} oder man stellt die Verbindung mithilfe eines Filehandlers her.
\end{frame}

\begin{frame}[fragile]{Die vollst\"andige Serverkommunikation}
	\lstinputlisting[linerange={3-3,19-34}]{resources/11_sendmail/mailer.py}
	\ \\[1cm]
	Auf \texttt{buildmessage()} wird im folgenden Teil eingegangen.
\end{frame}

\section{Komplexere Mails senden}
\subsection{Das Modul \texttt{email}}
\begin{frame}[fragile]{Das Modul \texttt{email}}
	Um mehr Möglichkeiten zur Gestaltung der E-Mail zu haben, lohnt sich die Verwendung des Moduls \texttt{email}.\\[.5cm]
	Mit \texttt{email.mime} lassen sich Emails individuell bauen und zusammensetzen. Außerdem kann man mehrteilige Mails und Mails mit Anhängen (z.B. Bildern) erstellen.
\end{frame}

\begin{frame}[fragile]{Die Klasse \texttt{MIMEText}}
	Die Klasse \texttt{email.mime.text.MIMEText()} erstellt ein MIME Objekt, welches hautps\"achlich aus Text besteht und einfach dem SMTP-Objekt \"ubergeben werden kann: \\ \ \\
	\lstinputlisting[lastline=1]{resources/11_sendmail/defs.py} \ \\
	
	\begin{description}
		\item[\_text] Ein String, der den Inhalt der Nachricht enth\"alt.
		\item[\_subtype] Der Untertyp des Objekts, per default \texttt{plain}
		\item[charset] Der Zeichensatz, der zur Kodierung der Zeichen verwendet werden soll. Standardm\"a\ss{}ig \texttt{us-ascii} oder \texttt{utf8}, abh\"angig von dem eingegebenen Text.
	\end{description}
\end{frame}

\begin{frame}[fragile]{Die Meta-Felder}
	Wenn das \texttt{MIMEText} Objekt instanziiert ist, muss dieses mit weiteren Informationen erg\"anzt werden:
	\lstinputlisting[firstline=10,lastline=15]{resources/11_sendmail/mailer.py}
	\begin{description}
		\item[From] Daten des Absenders (nur Mail oder Name und Mail)
		\item[To] Daten des Empfängers (nur Mail oder Name und Mail)
		\item[Subject] Betreff der Mail
	\end{description}
	Die Message l\"asst sich au\ss{}erdem noch um einen \texttt{Cc} oder einen \texttt{Bcc} erweitern.
\end{frame}

\begin{frame}{Das fertige Mail-Skript}
	\lstinputlisting[firstline=4,lastline=17]{resources/11_sendmail/mailer.py}
\end{frame}

% nothing to do from here on
\end{document}
