
%%	Getting started:
%%	1) Copy this file and name the copy it like the topic it covers
%%	2) In the following make sure to modify the copy, NOT THE ORIGINAL
%%	3) Look for the points marked with "TODO" and complete them
%%	4) compile this file

%% Nothing to modify here.
%% make sure to include this before anything else

\documentclass[german]{tudbeamer}
%\usetheme{Szeged}

% packages
\usepackage{color}
\usepackage{listings}

% color definitions
\definecolor{mygreen}{rgb}{0,0.6,0}
\definecolor{mygray}{rgb}{0.5,0.5,0.5}
\definecolor{mymauve}{rgb}{0.58,0,0.82}

% this is needed since the tudbeamer messes things up
\setbeamercolor{title}{fg=white}
\setbeamercolor{subtitle}{fg=white}
\setbeamercolor{supertitle}{fg=white}
\setbeamerfont{supertitle}{series=\bfseries,family=\sffamily}


% re-format the title frame page
\makeatletter
\def\supertitle#1{\gdef\@supertitle{#1}}%
\setbeamertemplate{title page}
{
  \vbox{}
  \vfill
  \begin{centering}
  \begin{beamercolorbox}[sep=8pt,center]{title}
      \usebeamerfont{supertitle}\@supertitle
   \end{beamercolorbox}
    \begin{beamercolorbox}[sep=8pt,center]{title}
    	\usebeamerfont{title}
    	\inserttitle\par%
      	\ifx\insertsubtitle\@empty%
     	\else%
        \vskip0.25em%
        {\usebeamerfont{subtitle}\usebeamercolor[fg]{subtitle}\insertsubtitle\par}%
      	\fi%     
    \end{beamercolorbox}%
    \vskip1em\par
    \begin{beamercolorbox}[sep=8pt,center]{author}
      \usebeamerfont{author}\insertauthor
    \end{beamercolorbox}
    \begin{beamercolorbox}[sep=8pt,center]{institute}
      \usebeamerfont{institute}\insertinstitute
    \end{beamercolorbox}
   \begin{beamercolorbox}[sep=8pt,center]{date}
      \usebeamerfont{date}\insertdate
    \end{beamercolorbox}\vskip0.5em
    {\usebeamercolor[fg]{titlegraphic}\inserttitlegraphic\par}
  \end{centering}
  \vfill
}
\makeatother


% insert frame number
%\expandafter\def\expandafter\insertshorttitle\expandafter{%
%      \insertshorttitle\hfill%
%\insertframenumber\,/\,\inserttotalframenumber}

% preset-listing options
\lstset{
  backgroundcolor=\color{white},   
  % choose the background color; 
  % you must add \usepackage{color} or \usepackage{xcolor}
  basicstyle=\footnotesize,        
  % the size of the fonts that are used for the code
  breakatwhitespace=false,         
  % sets if automatic breaks should only happen at whitespace
  breaklines=true,                 % sets automatic line breaking
  captionpos=b,                    % sets the caption-position to bottom
  commentstyle=\color{mygreen},    % comment style
  % deletekeywords={...},            
  % if you want to delete keywords from the given language
  extendedchars=true,              
  % lets you use non-ASCII characters; 
  % for 8-bits encodings only, does not work with UTF-8
  frame=single,                    % adds a frame around the code
  keepspaces=true,                 
  % keeps spaces in text, 
  % useful for keeping indentation of code 
  % (possibly needs columns=flexible)
  keywordstyle=\color{blue},       % keyword style
  % morekeywords={*,...},            
  % if you want to add more keywords to the set
  numbers=left,                    
  % where to put the line-numbers; possible values are (none, left, right)
  numbersep=5pt,                   
  % how far the line-numbers are from the code
  numberstyle=\tiny\color{mygray}, 
  % the style that is used for the line-numbers
  rulecolor=\color{black},         
  % if not set, the frame-color may be changed on line-breaks 
  % within not-black text (e.g. comments (green here))
  stepnumber=1,                    
  % the step between two line-numbers. 
  % If it's 1, each line will be numbered
  stringstyle=\color{mymauve},     % string literal style
  tabsize=4,                       % sets default tabsize to 4 spaces
  title=\lstname                   
  % show the filename of files included with \lstinputlisting; 
  % also try caption instead of title
}

% macro for code inclusion
\newcommand{\includecode}[2][c]{
	\lstinputlisting[caption=#2, style=custom#1]{#2}
}	% nothing to do here
% TODO change "course_info" to the name of your actual …_info(.tex)
%% Fill in metadata here that do not change over the course
%% They all are marked with the term "TODO".
%% Search functions usually do the trick

% TODO select the targeted language
% Select neither when using tudbeamer
%\usepackage[english]{babel}
\usepackage[ngerman]{babel}

% TODO select the encoding
\usepackage[utf8]{inputenc}
% usepackage[latin1]{inputenc}

\newcommand{\course}{
	Einführung in Python
}

\author{
	Felix Döring
}

\lstset{
	% TODO adapt these settings to your mainly used language
	% also see http://en.wikibooks.org/wiki/LaTeX/Source_Code_Listings
	% NOTE you can override these settings in individual cases
	language = Python,
	showspaces = false,
	showtabs = false,
	showstringspaces = false,
	escapechar = @
}

%% User defined macros here

% Does not work in tables! You have to use \lstinline$...$ instead!
\newcommand{\codeline}[1]{\colorbox{codegray}{\lstinline$#1$}}

% define my own colors
\definecolor{codegray}{gray}{0.97}
\definecolor{stringgreen}{rgb}{0.0, 0.7, 0.6}
 % TODO modify this if you have not already done so

% meta-information
\newcommand{\topic}{
	% TODO fill in the actual topic
	Regul\"are Ausdr\"ucke
}

% nothing to do here
\title{\topic}
\supertitle{\course}
\date{\today}

% the actual document
\begin{document}

\maketitle

\begin{frame}
	\tableofcontents
\end{frame}

\section{Grundlagen}
\begin{frame}{Grundlagen}
	Das \texttt{re} Modul der Python Standard Library ist die Python Implementierung von regulären Ausdrücken.\\[.25cm]
	Reguläre Ausdrücke werden verwendet um die Struktur von Text/Sprachen zu beschreiben.
\end{frame}

\begin{frame}{Grundlagen}
	Ein erstellter regul\"arer Ausdruck kann verwendet werden um:
	\begin{itemize}
		\item die Struktur eines Textes zu \"uberpr\"ufen
		\item bestimmte Teile eines Textes zu extraktieren
	\end{itemize}
\end{frame}

\begin{frame}{Grundlagen}
	Die Anwendung eines regulären Ausdruckes nennt man Matching.\\[.25cm]
Matcht der reguläre Ausdruck einem String, bedeutet dies, dass der String die Struktur hat, die der reguläre Ausdruck beschreibt.\\[.25cm]
Reguläre Ausdrücke werden oft auch 'regex' genannt (kurz für \textbf{regular expression}).
\end{frame}

\section{Matching Regeln}
\begin{frame}{Matching Regeln}
	Jeder Buchstabe und jede Zahl matcht immer \textbf{einmal} sich selbst.\\[.25cm]
	Das bedeutet:\\
	\begin{itemize}
		\item 	Der String \texttt{'a'} matcht der regex \texttt{'a'}.\\
		\item Der String \texttt{'abc'} matcht der regex \texttt{'abc'}, nicht aber regex \texttt{'a'}.\\
		\item \texttt{'4'} matcht \texttt{'4'}, nicht aber \texttt{'5'}, \texttt{'a'} oder \texttt{'41'} usw.
	\end{itemize}
\end{frame}

\subsection{Sonderzeichen}
\begin{frame}{Sonderzeichen}
	\texttt{.} (Punkt) matcht \textbf{einem} beliebingen Schriftzeichen, außer \texttt{'\textbackslash{}n'} (newline)\\[.25cm]
	Für die regex `'.'` gilt:\\
	\begin{itemize}
		\item \texttt{'a'} matcht
		\item \texttt{'b'} matcht
		\item \texttt{'4'} matcht
		\item \texttt{'ab'} matcht nicht, denn \texttt{'.'} ist nur ein Zeichen und \texttt{'ab'} sind zwei.
	\end{itemize}
\end{frame}

\begin{frame}{Sonderzeichen}
	\texttt{[]} matcht jedem der in den Klammern stehenden Zeichen, jedoch nur \textbf{einmal} (wie bei \texttt{.}).\\[.25cm]
	Für die regex `'[abg]'` gilt also:\\
	\begin{itemize}
		\item \texttt{'a'} matcht
		\item \texttt{'b'} matcht
		\item \texttt{'g'} matcht
		\item \texttt{'ab'} matcht nicht (zwei zeichen matchen nicht einem).
	\end{itemize}
\end{frame}

\begin{frame}{Sonderzeichen}
	\texttt{()} erstellt eine Gruppe. Alles was in den Klammern steht, muss genau so vor kommen.\\[.25cm]
	Für die regex `'(abc)'` gilt:\\
	\begin{itemize}
		\item \texttt{'a'} matcht nicht
		\item \texttt{'ab'} matcht nicht
		\item \texttt{'abc'} matcht
	\end{itemize}
\end{frame}

\begin{frame}{Sonderzeichen}
	\texttt{\textbackslash} escaped ein Sonderzeichen.\\[.25cm]
	Alle hier aufgeführten Sonderzeichen können nicht in einem Pattern vorkommen, als das, was sie eigentlich bedeuten, dafür müssen sie extra markiert werden.\\
	\begin{itemize}
		\item \texttt{'\textbackslash\textbackslash'} als Pattern matcht auf den String \texttt{'\textbackslash'}
		\item \texttt{'\textbackslash.'} als Pattern matcht auf den String \texttt{'.'}
	\end{itemize}
\end{frame}

\begin{frame}{Sonderzeichen}
	\texttt{\^} matcht ab dem Anfang eines Strings oder ab jedem \texttt{\textbackslash{}n}\\[.25cm]
	Für die regex \texttt{'\^{}a'} ergibt sich also:\\
	\begin{itemize}
		\item \texttt{'a'} matcht
		\item \texttt{'ba'} matcht nicht, da der falsche Character am Anfang steht.
		\item \texttt{'aba'} matcht
	\end{itemize}	
\end{frame}

\begin{frame}{Sonderzeichen}
	\texttt{\$} matcht dem Ende eines Strings (oder dem Zeilenende)\\[.25cm]
	Für die regex \texttt{'a\$'} folgt daraus:\\
	\begin{itemize}
		\item \texttt{'a'} matcht
		\item \texttt{'ba'} matcht
		\item \texttt{'bab'} matcht nicht, da der falsche Character am Ende steht
		\item \texttt{'aba'} matcht
	\end{itemize}
\end{frame}

\begin{frame}{Sonderzeichen}
	\texttt{|} ist ein \textbf{ODER}. Entweder die regex davor oder danach muss matchen.\\[.25cm]
	Für die regex `'a|b'` gilt:\\
	\begin{itemize}
		\item \texttt{'a'} matcht
		\item \texttt{'b'} matcht
		\item \texttt{'ab'} matcht nicht, da \texttt{'a|b'} mit [ab] gleichzusetzen ist
	\end{itemize}
\end{frame}

\subsection{Zusammengesetzte Regex}
\begin{frame}{Zusammengesetzte Regex}
	Regular Expressions setzen sich aus kleineren Regular Expressions zusammen.\\
	So kann man z.B. auch festlegen, wie häufig ein Zeichen auftreten soll.
\end{frame}

\begin{frame}{Zusammengesetzte Regex}
	\texttt{*} - Die \textit{vorangestellte} Regex muss 0 - n Mal vorkommen.\\[.25cm]
	Für die regex `'a*'` gilt:\\
	\begin{itemize}
		\item \texttt{''} matcht
		\item \texttt{'a'} matcht
		\item \texttt{'aa'} matcht
		\item \texttt{'aaaab'} matcht nicht, da ein anderes Zeichen als 'a' vorkommt
	\end{itemize}
\end{frame}

\begin{frame}{Zusammengesetzte Regex}
	\texttt{+} - Die \textit{vorangestellte} Regex muss 1 - n Mal vorkommen.\\[.25cm]
	Für die regex `'a+'` gilt:\\
	\begin{itemize}
		\item \texttt{''} matcht nicht, da es kein mal vorkommt
		\item \texttt{'a'} matcht
		\item \texttt{'aa'} matcht
		\item \texttt{'ab'} matcht nicht, da ein anderes Zeichen als 'a' vorkommt
	\end{itemize}
\end{frame}

\begin{frame}{Zusammengesetzte Regex}
	\texttt{?} - Die \textit{vorangestellte} regex muss 0 - 1 Mal vorkommen.\\[.25cm]
	Für die regex `'a?'` gilt:\\
	\begin{itemize}
		\item \texttt{''} matcht
		\item \texttt{'a'} matcht
		\item \texttt{'aa'} matcht nicht, da das Zeichen öfter, als nur ein mal vorkommt
	\end{itemize}
\end{frame}

\begin{frame}{Zusammengesetzte Regex}
	\texttt{\{m\}} - Die \textit{vorangestellte} regex muss genau m Mal vorkommen.\\[.25cm]
	Für die regex `'y\{3\}'` gilt:\\
	\begin{itemize}
		\item \texttt{'yyy'} matcht
		\item \texttt{'y'} matcht nicht, da es zu wenige Zeichen sind
		\item \texttt{'yyyy'} matcht nicht, da es mehr Zeichen sind
	\end{itemize}
\end{frame}

\begin{frame}{Zusammengesetzte Regex}
	\texttt{\{m,n\}} -  Die \textit{vorangestellte} regex muss m - n Mal vorkommen.\\[.25cm]
	Für die regex `'y\{2,5\}'` gilt:\\
	\begin{itemize}
		\item \texttt{'yyy'} matcht
		\item \texttt{'y'} matcht nicht, da es zu wenige Zeichen sind
		\item \texttt{'yyyy'} matcht
		\item \texttt{'yyyyyy'} matcht nicht, da es zu viele Zeichen sind
	\end{itemize}
\end{frame}

\subsection{Spezielle Sequenzen}
\begin{frame}{Spezielle Sequenzen}
	Desweiteren gibt es noch spezielle Sequenzen, wie z. B.\\[.25cm]
	\begin{itemize}
		\item \texttt{\textbackslash{}d} für Unicode Ziffern, äquivalent für \texttt{[0-9]}
		\item \texttt{\textbackslash{}D} ist das Gegenteil, alles was keine Unicode Ziffern sind
		\item \texttt{\textbackslash{}s} für alle Whitespace Zeichen, das entspricht \texttt{'[\ \textbackslash{}t\textbackslash{}n\textbackslash{}r\textbackslash{}f\textbackslash{}v]'}
		\item \texttt{\textbackslash{}S} entspricht wieder dem Gegenteil
		\item \texttt{\textbackslash{}w} für alle Unicode Zeichen \texttt{'[a-zA-Z0-9\_]'}
		\item \texttt{\textbackslash{}W} für alle Nicht-Unicode Zeichen
		\item \texttt{[\^{}...]} entspricht allem, was nicht in den Klammern steht
	\end{itemize}
\end{frame}

\section{Methoden}
\begin{frame}[fragile]{Methoden}
	\lstinputlisting[lastline=1]{resources/10_regex/methods.py}
	Wandelt einen String in ein regular expression Objekt um.\\[.25cm]
	\lstinputlisting[firstline=3, lastline=3]{resources/10_regex/methods.py}
	Sucht in `string` nach dem Pattern `pattern`.\\[.25cm]
	\lstinputlisting[firstline=5, lastline=5]{resources/10_regex/methods.py}
	Sucht am Begin des Strings nach dem Pattern.
\end{frame}

\begin{frame}[fragile]{Methoden}
	\lstinputlisting[firstline=7, lastline=7]{resources/10_regex/methods.py}
	Der komplette String und das Pattern müssen übereinstimmen.\\[.25cm]
	\lstinputlisting[firstline=9, lastline=9]{resources/10_regex/methods.py}
	Gibt eine Liste von Strings mit allen passenden Übereinstimmungen zurück.\\[.25cm]
	\lstinputlisting[firstline=11, lastline=11]{resources/10_regex/methods.py}
	Gibt einen Iterator, welcher `match` Objekte beinhaltet zurück\\[.25cm]
	\textbf{\href{https://docs.python.org/3/library/re.html}{Die restlichen Funktionen können in den Docs gefunden werden.}}
\end{frame}

\section{reqular expression Objekt}
\begin{frame}{reqular expression Objekt}
	Ein solches Objekt hat im Großen und Ganzen die selben Methoden, jedoch ohne zusätzliches Pattern, da das Objekt an sich bereits ein Pattern enthält.
\end{frame}

\section{Match Objekt}
\begin{frame}{Match Objekt}
	\lstinputlisting[lastline=1]{resources/10_regex/match.py}
	Gibt die Startposition des Patterns im String zurück.\\[.25cm]
	\lstinputlisting[firstline=3, lastline=3]{resources/10_regex/match.py}
	Gibt die Endposition des Patterns im String zurück.\\[.25cm]
	\lstinputlisting[firstline=5, lastline=5]{resources/10_regex/match.py}
	Gibt ein Tuple zurück \texttt{(m.start([group]), m.end([group]))}
\end{frame}


% nothing to do from here on
\end{document}
