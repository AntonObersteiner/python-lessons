
%%	Getting started:
%%	1) Copy this file and name the copy it like the topic it covers
%%	2) In the following make sure to modify the copy, NOT THE ORIGINAL
%%	3) Look for the points marked with "TODO" and complete them
%%	4) compile this file

%% Nothing to modify here.
%% make sure to include this before anything else

\documentclass[german]{tudbeamer}
%\usetheme{Szeged}

% packages
\usepackage{color}
\usepackage{listings}

% color definitions
\definecolor{mygreen}{rgb}{0,0.6,0}
\definecolor{mygray}{rgb}{0.5,0.5,0.5}
\definecolor{mymauve}{rgb}{0.58,0,0.82}

% this is needed since the tudbeamer messes things up
\setbeamercolor{title}{fg=white}
\setbeamercolor{subtitle}{fg=white}
\setbeamercolor{supertitle}{fg=white}
\setbeamerfont{supertitle}{series=\bfseries,family=\sffamily}


% re-format the title frame page
\makeatletter
\def\supertitle#1{\gdef\@supertitle{#1}}%
\setbeamertemplate{title page}
{
  \vbox{}
  \vfill
  \begin{centering}
  \begin{beamercolorbox}[sep=8pt,center]{title}
      \usebeamerfont{supertitle}\@supertitle
   \end{beamercolorbox}
    \begin{beamercolorbox}[sep=8pt,center]{title}
    	\usebeamerfont{title}
    	\inserttitle\par%
      	\ifx\insertsubtitle\@empty%
     	\else%
        \vskip0.25em%
        {\usebeamerfont{subtitle}\usebeamercolor[fg]{subtitle}\insertsubtitle\par}%
      	\fi%     
    \end{beamercolorbox}%
    \vskip1em\par
    \begin{beamercolorbox}[sep=8pt,center]{author}
      \usebeamerfont{author}\insertauthor
    \end{beamercolorbox}
    \begin{beamercolorbox}[sep=8pt,center]{institute}
      \usebeamerfont{institute}\insertinstitute
    \end{beamercolorbox}
   \begin{beamercolorbox}[sep=8pt,center]{date}
      \usebeamerfont{date}\insertdate
    \end{beamercolorbox}\vskip0.5em
    {\usebeamercolor[fg]{titlegraphic}\inserttitlegraphic\par}
  \end{centering}
  \vfill
}
\makeatother


% insert frame number
%\expandafter\def\expandafter\insertshorttitle\expandafter{%
%      \insertshorttitle\hfill%
%\insertframenumber\,/\,\inserttotalframenumber}

% preset-listing options
\lstset{
  backgroundcolor=\color{white},   
  % choose the background color; 
  % you must add \usepackage{color} or \usepackage{xcolor}
  basicstyle=\footnotesize,        
  % the size of the fonts that are used for the code
  breakatwhitespace=false,         
  % sets if automatic breaks should only happen at whitespace
  breaklines=true,                 % sets automatic line breaking
  captionpos=b,                    % sets the caption-position to bottom
  commentstyle=\color{mygreen},    % comment style
  % deletekeywords={...},            
  % if you want to delete keywords from the given language
  extendedchars=true,              
  % lets you use non-ASCII characters; 
  % for 8-bits encodings only, does not work with UTF-8
  frame=single,                    % adds a frame around the code
  keepspaces=true,                 
  % keeps spaces in text, 
  % useful for keeping indentation of code 
  % (possibly needs columns=flexible)
  keywordstyle=\color{blue},       % keyword style
  % morekeywords={*,...},            
  % if you want to add more keywords to the set
  numbers=left,                    
  % where to put the line-numbers; possible values are (none, left, right)
  numbersep=5pt,                   
  % how far the line-numbers are from the code
  numberstyle=\tiny\color{mygray}, 
  % the style that is used for the line-numbers
  rulecolor=\color{black},         
  % if not set, the frame-color may be changed on line-breaks 
  % within not-black text (e.g. comments (green here))
  stepnumber=1,                    
  % the step between two line-numbers. 
  % If it's 1, each line will be numbered
  stringstyle=\color{mymauve},     % string literal style
  tabsize=4,                       % sets default tabsize to 4 spaces
  title=\lstname                   
  % show the filename of files included with \lstinputlisting; 
  % also try caption instead of title
}

% macro for code inclusion
\newcommand{\includecode}[2][c]{
	\lstinputlisting[caption=#2, style=custom#1]{#2}
}	 % nothing to do here
% TODO change "course_info" to the name of your actual …_info(.tex)
%% Fill in metadata here that do not change over the course
%% They all are marked with the term "TODO".
%% Search functions usually do the trick

% TODO select the targeted language
% Select neither when using tudbeamer
%\usepackage[english]{babel}
\usepackage[ngerman]{babel}

% TODO select the encoding
\usepackage[utf8]{inputenc}
% usepackage[latin1]{inputenc}

\newcommand{\course}{
	Einführung in Python
}

\author{
	Felix Döring
}

\lstset{
	% TODO adapt these settings to your mainly used language
	% also see http://en.wikibooks.org/wiki/LaTeX/Source_Code_Listings
	% NOTE you can override these settings in individual cases
	language = Python,
	showspaces = false,
	showtabs = false,
	showstringspaces = false,
	escapechar = @
}

%% User defined macros here

% Does not work in tables! You have to use \lstinline$...$ instead!
\newcommand{\codeline}[1]{\colorbox{codegray}{\lstinline$#1$}}

% define my own colors
\definecolor{codegray}{gray}{0.97}
\definecolor{stringgreen}{rgb}{0.0, 0.7, 0.6}
 % TODO modify this if you have not already done so
\usepackage{comment}

% meta-information
\newcommand{\topic}{
	% TODO fill in the actual topic
	Module und Pakete
}

% nothing to do here
\title{\topic}
\supertitle{\course}
\date{\today}

% the actual document
\begin{document}

\maketitle

\begin{frame}
	\tableofcontents
\end{frame}

%% Für einzelne Zeilen Code \codeline{} vewenden (ist mit Markup) für einzelne Worte \texttt{} verwenden (auch in Tabellen!)

\section{Modules}
\subsection{Eigene Module}
\begin{frame}[fragile]{Module}
	\begin{itemize}
		\item Ein Module ist die python-interne Repräsentation einer \texttt{.py} Datei
		\item Der Dateiname setzt sich daher aus Modulname + \texttt{.py} zusammen
		\item Module beinhalten eine beliebige Anzahl an Definitionen (Klassen, Funktionen, Variablen/Konstanten)
		\item Das komplette Modul wird mit \texttt{import} hinzugefügt
		\item Einzelne Inhalte mit \texttt{from} Modul \texttt{import} Name
		\item Mit Hilfe von \texttt{as} kann ein Alias für den importierten namen angelegt werden
	\end{itemize}

\end{frame}

\begin{frame}
	\lstinputlisting{resources/04_modules_and_packages/incdec.py}
	\lstinputlisting{resources/04_modules_and_packages/modules.py}
\end{frame}

\subsection{Suchpfad für Module}
\begin{frame}
	Wenn ein Modul importiert wird, sucht Python nach diesem an allen Orten, welche im sogenannten PYTHONPATH aufgelistet sind. Python intern ist dieser in \texttt{sys.path} zu finden, in der Kommandozeile ist er mit der Umgebungsvariable PYTHONPATH erreichbar.\\
	Dieser Pfad \texttt{sys.path} kann zur Laufzeit verändert werden.\\
	Standardmä\ss{}ig enthält er meist:
	\begin{itemize}
		\item die Standard Bibliothek der derzeitig verwendeten Python Version
		\item das aktuellen Verzeichnis in dem der Interpreter aufgerufen wurde
		\item eine Verzeichnis mit plattfomspezifischen Modulen, z.B.
			"/usr/local/Cellar/python3/3.4.3/Frameworks/Python.framework/
			Versions/3.4/lib/python3.4/plat-darwin" für Mac
		\item Benutzerspezifische Module für die jeweilige Python Version, z.B. "/usr/local/lib/python3.4/site-packages" für Python 3.4
	\end{itemize}
\end{frame}

\subsection{Suchpfad modifizieren}
\begin{frame}
Zusätzliche directories für den PYTHONPATH kann man beim aufruf übergeben\\
\texttt{PYTHONPATH=/my/directory python3 script.py}\\
Oder im programm
\lstinputlisting{resources/04_modules_and_packages/path.py}
Änderungen am pfad sind erst \textbf{nach} der entsprechenden Codezeile verfürbar.
\end{frame}

\subsection{Standart Module}
\begin{frame}{Standart Module}
	Python liefert viele nützliche Module bereits in der Standardbibliothek mit, von welchen wir eineige bereits kennen gelernt haben.
	z.B. sys, os, http, re, fuctools, itertools, collections, hashlib, urllib und viele mehr. Daher bezeichnet man Python oft auch als "Batteries included".
\end{frame}

\section{Packages}
\begin{frame}{Packages}
	\begin{itemize}
		\item Packages sind Ordner, welche mindestens ein \texttt{\_\_init\_\_.py} Modul enthalten.
		\item der Inhalt dieses Moduls ist prinzipiell egal.
		\item Packages können genau wie Module importiert werden.
		\item Wird das Package selbst importiert, sind alle Definitionen in \texttt{\_\_init\_\_.py} über das importierte Package erreichbar.
		\item Sie werden benutzt um Module in sinnvolle Gruppen zusammenzufassen.
	\end{itemize}
\end{frame}

\end{document}
